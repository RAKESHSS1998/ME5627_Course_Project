%************************************************
\chapter{Introduction}\label{ch:introduction}
%************************************************
The study of fluid flow around bluff bodies, such as cylindrical structures, is an essential aspect of engineering and scientific research due to its widespread implications in both natural and industrial contexts. One key phenomenon in this field is vortex shedding, where alternating vortices are generated in the wake of a cylindrical object as fluid flows over it. This effect can result in periodic forces acting on the structures, which can cause vibrations that affect their stability, lifespan, and performance. Such vortex-induced vibrations are particularly significant for applications involving bridges, tall buildings, towers, pipelines, and offshore structures. Understanding the conditions under which vortex shedding occurs and how it behaves is crucial for designing structures that can withstand these dynamic forces without failure.
To explore this phenomenon in a controlled environment, this project focuses on the design, fabrication, and testing of a wind tunnel tailored for investigating vortex shedding around cylindrical objects. The wind tunnel is designed to facilitate consistent airflow conditions for various Reynolds numbers, thus enabling the study of changes in flow velocity which affects the vortex shedding process. In addition, a constant current hotwire anemometer is utilized to accurately measure vortex shedding frequencies, while smoke visualization is used for qualitative analysis of vortex formation and behavior. This project aims to connect experimental observations with theoretical concepts, enhancing the understanding of vortex shedding and contributing to advances in fluid mechanics research.


\section{Objectives:}

\begin{enumerate}

\item Design a square cross-section wind tunnel capable of generating controlled airflow over a cylindrical test object, ensuring steady flow conditions.

\item Visualize vortex shedding and flow patterns around the cylindrical test object using smoke.

\item Measure airflow velocity and hence vortex shedding frequency using a hot wire anemometer and tungsten bulb filament, ensuring accurate data capture in turbulent flow regime.

\item Compare velocity data from both measurement techniques to assess their consistency, accuracy, and reliability under varying velocity flow conditions.

\item Analyze the relationship between flow velocity, vortex formation, and flow characteristics (e.g., turbulence) around the test object.

\item Analyze the results and plot St (Strouhal number) versus Re (Reynolds number) to understand the behavior of vortex shedding at various flow conditions.

\end{enumerate}

\section{Organization of the report}
This project report has three parts.
\begin{itemize}
\item Part~\ref{pt:intro} contains the introduction and the literature review chapters.
\item Part~\ref{pt:theory} includes Chapters~\ref{ch:design}, \ref{ch:procedure} and \ref{ch:results}:
	\begin{enumerate}
	\item[-] Chapter~\ref{ch:design} discusses the design parameters of the wind tunnel - size and shape of convergent, divergent and test sections. It also discusses the design of the hot wire anemometer
	\item[-] Chapter~\ref{ch:procedure} discusses calibration procedure of both the tungsten filament of the bulb and the hot wire anemometer. The chapter also discusses the time constant and the selection of sensor for this project.
    \item[-] Chapter~\ref{ch:results} discusses the behavior of vortex in the wind tunnel at different locations. In addition, it discusses the relation between the Reynolds number and the Strouhal number at those locations
    \end{enumerate}
\item Part~\ref{pt:conclusion} briefs the conclusions on the work presented in this report.
\end{itemize}






%*****************************************
%*****************************************
%*****************************************
%*****************************************
%*****************************************
