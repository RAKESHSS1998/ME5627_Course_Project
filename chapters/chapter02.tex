%*****************************************
\chapter{Literature review}\label{ch:Literature}
%*****************************************
%\setcounter{figure}{10}
% \begin{flushright}
% \itshape Robert Cialdini, Scott Adams, and Tony Robbins
% \end{flushright}
% \NoCaseChange{Homo Sapiens}


The study by Doolan \textit{et al}.~\cite{DOOLAN2012} focuses on the behavior of vortex shedding in the wake of a cylinder subjected to flow through a gap, with a particular emphasis on how the gap-to-diameter ratio (denoted as G/D) influences the spectral properties of the wake. The paper provides valuable insights into the interaction between shear layers and the formation of quasi-stable vortices in the near wake, a critical phenomenon in fluid dynamics, particularly in the context of flow-induced vibrations, aero acoustics, and the design of cylindrical structures in flow environments.

Hauptmann \textit{et al}.~\cite{UHartmann_1982} developed an affordable and robust hot wire anemometer using a filament from a flashlight bulb, which makes it suitable for educational and practical applications despite a slight reduction in precision due to filament looping. This design addressed the fragility and high cost of traditional tungsten or platinum probes, allowing for a wider use in turbulence and fluid flow studies. Similarly, Griffith \textit{et al}.~\cite{HOSSEINI2020103103} and Singha and Sinhamahapatra (2010) observed an increase in St with confinement, although Turki et al. (2003) noted a decreasing trend. For vertical confinement (h), studies such as Inoue and Sakuragi (2008) reported an increase in both St and critical Re as h decreased.

Ishan J. Kelkar \textit{ et al.}.~\cite{kelkar2019design}. This article incorporates a discussion on arranging and gathering an unassuming wind current that can be utilized for instructive purposes. The objective was to make a satisfactory wind stream at a speed of 10 m/s in the test fragment and to incorporate factors like traditionalism, comfort, and robustness into the development. Another objective was to direct an investigation on comparable proliferation and show the differentiation among numerical and exploratory outcomes.

Amol L. Mangrulkar\textit{ et al.}.~\cite{mangrulkar2019design} discuss the design and assembly of a simple wind tunnel suitable for educational purposes. The goal was to achieve adequate airflow with a speed exceeding 10 m/s in the test section, incorporating factors such as simplicity, comfort, and durability in its construction. Another objective was to conduct an investigation on vortex shedding and compare the results between numerical simulations and experimental data.

3D experimental studies by Rehimi\textit{ et al.}~\cite{rehimi} and Jung \textit{et al}. (2012) extended the analysis to confined 3D flows, identifying trends in recirculation length and wake dynamics, while time-resolved data capturing transient vortex-shedding in microscale flows remain scarce due to the limitations of conventional micro PIV systems.

Sahin and Owens~\cite{SAHIN2004121} demonstrated that increasing lateral confinement (reducing w) raises the critical Re for vortex-shedding. 2D simulations of lateral confinement, as studied by Singha and Sinhamahapatra \textit.~\cite{SINGHA2010757}, provided valuable insight into the trends of vortex-shedding frequency.







%*****************************************
%*****************************************
%*****************************************
%*****************************************
%*****************************************
