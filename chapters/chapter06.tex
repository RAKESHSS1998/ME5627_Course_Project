\chapter{Conclusions}\label{ch:conclusions}
This project successfully investigated vortex shedding around a cylindrical body using a custom designed and fabricated wind tunnel under controlled airflow conditions. By analyzing Reynolds numbers (Re) in the range of 2699 to 5399 and the corresponding Strouhal numbers (St) between 0.2 and 0.6, the study established the relationship between flow velocity and vortex shedding characteristics. The peak vortex shedding frequencies ranged from 0.63 Hz to 37.06 Hz across different flow velocities and measurement locations, highlighting the dynamic nature of flow interactions with the solid cylinder. 

A significant outcome of the project is the successful calibration and implementation of a custom-built hot wire anemometer, with constants B = 0.676 and n = 0.232 (in King's law), for precise airflow velocity measurements. The hot wire anemometer achieved a time constant of 6 ms, significantly faster compared to the tungsten filament sensor, which had a time constant of 450 ms. This considerable difference underscores the superior responsiveness of the hot-wire anemometer to capture rapid flow fluctuations. Additionally, smoke visualization vividly illustrated the formation and shedding of vortices behind the cylinder, complementing the quantitative measurements.

The results showed a clear trend in vortex shedding behavior, with Strouhal numbers decreasing slightly as Reynolds numbers increased, indicating a lower vortex frequency at higher velocities. This trend was most evident in regions of transitioning turbulence, offering valuable insights into the interplay between inertial and viscous forces in fluid flow. The study also revealed variations in the vortex shedding frequencies at different measurement heights from the center of the solid cylinder, emphasizing the spatial complexity of flow dynamics.

The findings enhance the understanding of fluid dynamics, particularly the relationship between flow velocity, vortex frequency, and turbulence. This research provides a solid foundation for future studies in advanced sensing techniques, computational fluid dynamics simulations, and structural designs that address vortex-induced vibrations. It highlights the importance of combining experimental and theoretical approaches to advance research in fluid mechanics.

\section{Future scope}

Testing different geometries or adding varying surface roughness to cylindrical objects in future studies, combined with advanced sensing techniques, could help identify how vortex shedding behaviors change under different physical conditions.

Piezoelectric and piezoresistive sensors can be used for more comprehensive flow and pressure measurements. Piezoelectric sensors can capture dynamic pressure fluctuations, while piezoresistive sensors can offer more stable, continuous measurements of strain and pressure, providing a broader range of data for vortex analysis. The data gathered from both piezoelectric and piezoresistive sensors could be compared with computational fluid dynamics (CFD) simulations for validation, helping to refine the experimental results and improve the understanding of vortex shedding and flow characteristics at various Reynolds numbers.

The hot wire anemometer can be improved by integrating modern data acquisition systems with higher sampling rates and reduced noise interference. Additionally, laser-based techniques such as Particle Image Velocimetry (PIV) could provide more detailed velocity field visualizations.Scaled-down models of industrial and natural structures can be tested in the wind tunnel to simulate real-world scenarios. This approach can help optimize designs and improve the resilience of structures subjected to vortex-induced forces.

Insights from vortex shedding studies can be applied to the design of efficient energy systems, such as wind turbines and vortex-induced energy harvesters. Further, studying pollutant dispersion in wakes could have implications for environmental modeling and management. Extending the research to higher Reynolds numbers could uncover nonlinear flow behaviors and transitional phenomena, providing a deeper understanding of turbulence and wake dynamics in fluid systems.


%---------------------------- Bibliography -------------------------------

% Please add the contents of the .bbl file that you generate,  or add bibitem entries manually if you like.
% The entries should be in alphabetical order

